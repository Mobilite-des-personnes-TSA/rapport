\section{Conclusion}


The goal of this project was to develop a routing application that would take into account issues encountered by autistic people. Thanks to the Tisseo API, we were able to gather information about Toulouse's public transport network. Then, we implemented an A-Star algorithm to provide the user with the route that features as little nuisance as possible with additional details on issues the user may encounter and precautions they may take. The application takes into account real-time events to adapt the suggested path. We also provided an adapted interface to make the application easier to use for people with ASD.


However, the application still relies on static data for luminosity and noise, and the extensive use of the Tisseo API restricts the range of use to the city of Toulouse. The next version of the application should focus on gathering information about luminosity and noise from live data. This will improve the adaptability of the routing algorithm and provide better paths to users that are sensitive to those factors. Future works should also implement a routing algorithm inside the application. This will make the application less dependent on the Tisseo API and will be a first step to deploying it in other cities. Finally, future versions should take into account the feedback of the user to adapt the A Star heuristic.


This project allowed us to discover the development of mobile apps with Kotlin and Android, which is a necessity as most web and desktop applications have a mobile version. Additionally, this project made us consider the accessibility of our application and techniques to ensure everyone can use it. It also granted us the opportunity to use GitHub tools for project handling, which proved to be very useful combined with the agile method.
