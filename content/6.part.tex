\section{Adapted interfaces}

Since our application aims to help people with ASD, interface and the display of information are an important part of our project. Uncertainty and service disruptions are sources of anxiety for people with ASD when using public transports \cite{2020ExperiencesYoungAutistic}, so information should be brought to the user in a clear format and with no room for doubt.\newline

In fact, previous research revealed that people with ASD and their caregivers rely a lot on images to share and receive information \cite{2018MobilityPoliciesExtraSmall}. Taking this into account, we decided to use pictograms to label areas which may prove to be stressful to people with ASD. These pictures represent situation that the user may have already experienced and know to be unpleasant, allowing them to anticipate this issue or avoid it completely. Additionally, pictograms are also used to suggest devices or other objects that can help them feel comfortable during their commutes.\newline 

Finally, we conceived the journey planner as simple as possible, to not give too much information at the same time to the user and remove any uncertainty. The Tisseo API usually returns different possible paths to travel between destinations, but our application chooses the best out of all the available options and returns only one path to the user. The Tisseo API also allows our app to line up with public transports schedule, and this information can be relayed to the user so that they know how long they will stay at one location.
