\sectionnn{Conclusion}

In conclusion, this state-of-the-art review highlighted the multiple difficulties faced by people with autism spectrum disorders (ASD) in navigating public transport systems. Whether it's increased anxiety, sensory sensitivities, dependence on healthcare providers or limited accessibility, the barriers faced by this part of the population underline the urgent need for innovative solutions to support inclusion and autonomy.

Existing accommodations, such as careful route planning and the use of support networks, offer partial relief but often come at the detriment of independence. What's more, while emerging technologies are promising for meeting specific needs, they are not without their limitations. From problems of compatibility with existing infrastructures to concerns about the imposition of neurotypical norms, the path to truly inclusive mobility requires a nuanced and comprehensive approach.

The solutions proposed, ranging from sensory-friendly urban design to wearable assistive technologies, offer a glimpse of a more inclusive future. However, their effectiveness depends on resolving key issues, such as resource limitations, infrastructure compatibility and the imperative of empowering people with ASD rather than imposing compliance.

In addition, the importance of early diagnosis and intervention is a crucial aspect that deserves attention. By taking advantage of innovative technologies, such as virtual reality-based diagnostic tools, we can identify ASD at earlier stages, enabling targeted support and interventions that can have a positive impact on individuals' ability to navigate the world around them.

As we move towards a more inclusive society, it is imperative to prioritize the voices and needs of people with ASD, ensuring that solutions are not only adaptive, but also empowering. By encouraging collaboration between researchers, policymakers and the autistic community, it is possible to harness the transformative potential of technology to create transportation systems that effectively serve all members of society, regardless of neurodiversity.
\medskip



