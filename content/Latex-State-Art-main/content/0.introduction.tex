\sectionnn{Introduction}

Nowadays, with the increasing growth of populations, mobility has become a constant axis of development. Cities put more and more effort in expanding their public transportation networks in order to serve the consistent flow of users and avoid traffic jams. Additionally, mobility must adapt to the challenges of the XXIst century like ecology, hence the development of alternate ways to get about.

However, mobility still fails to address the issue of inclusivity. An increasing population means an increasing number of people with disabilities or disorders and some infrastructures are still not adapted to the special needs they may have. Modern constructions have started to evolve, specifically to be more accessible for deaf people, blind people, or wheelchair users.

But one of the disorders that stays under the radar when it comes to public transportation policies is Autism Spectrum Disorder (ASD). Given that it is an invisible disability, it usually goes unnoticed, even though it may cause numerous issues when using vehicles or public transportation. These issues should be investigated in order to provide solutions for the mobility of autistic people.

This document aims to understand the challenges regarding the mobility of the autistic community. First, we define what ASD is and what problems this community faces. Next, we focus on the specific issues regarding their mobility. Then, we take a detailed look at the existing solutions for autistic people. Finally, we analyze their limits and explore how future efforts should address these shortcomings.
