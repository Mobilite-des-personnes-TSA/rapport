\section{Methodology and tools used}

\subsection{Android studio with kotlin}
To develop our application, we decided to use Android Studio and Kotlin because they offer many advantages in Android development. Indeed, Kotlin is an efficient programming language for building Android applications. 
Kotlin has a concise syntax and interesting features, like intelligent casts and extension functions that reduces the amount of code to write.
Kotlin also offers support for functional programming concepts such as lambda functions and extension functions. It grants us the right to write clean and modular code easily. Kotlin is also well supported by Android Studio, the Integrated Development Environment (IDE) we have used to develop our application.\newline

Android Studio was our choice of IDE. It gives us an great programming tool, while allowing us to emulate easily smartphones to test on those devices the interfaces and functionalities of our application. \newline

In order to structure our project, we created various activities for each part of our project. It is divided into three parts. First of all, we created an activity for the user to describe their needs. For that, we created a User Interface (UI), named “general settings”. For instance, in this UI, they can tell whether they are sensitive to noise, light or crows. We will therefore decide an itinerary that is adapted to their needs. Then, we created a second activity that allows the user to choose the points of departure, arrival, but also at which time they want to leave, and the transportation mode that they want to use. All this information is used in the main activity that gives the best itinerary to the user regarding all their personal needs. \newline


\subsection{Databases and API}
In order to generate the map, we used the API OpenStreetMap (OSM). It is a collaborative mapping project that provides a free, editable map of the world. In our project, we used OSM data within Kotlin and Android Studio to generate a map of Toulouse and calculate routes between two points. By integrating OSM's API into our application, we have been able to access detailed geographical information, including roads, landmarks, and points of interest, allowing us to generate an interactive map interface within our app. Thanks to OSM, we also implemented the route calculation functionality, enabling users to input their desired start and end points and receive optimal route suggestions based on real-time OSM data. This integration allowed us to have a first itinerary without considering the user’s needs and the public transportation. \newline

Since we want to use public transportation for our application, we needed another tool than OSM to generate our routes.This is why we also integrated the Tisséo API for generating routes with public transportation in Toulouse. This API gave us access to all the data of the Tisséo transportation network, including all the buses, metros and tramways. We also had access to the public transportation timetable. This integration enabled us to generate a route at a certain time, using a combination of walking and public transportation in the city of Toulouse. By combining OSM's detailed geographical data with Tisséo's public transportation information, our application offered users a comprehensive and efficient navigation solution tailored to the unique transportation infrastructure of Toulouse. \newline


