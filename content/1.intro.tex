\section{Introduction}

Nowadays, with the increasing growth in population, the ability to move around has become an important factor to live as an integrated member of society. However, mobility fails to address the issue of inclusivity. Our current infrastructures are not built to address the needs of people with disabilities. More precisely, individuals with Autism Spectrum Disorder (ASD), still face multiple challenges due to the inadequacy and overstimulation caused during their daily commutes.

Although this is a well-defined field of research, this paper should be seen as a contribution to the conversation that aims at understanding disabilities as a product of processes of human-environment interaction, and an attempt to address mobility issues faced by people with disabilities by taking into account their challenges and capabilities.

As we have seen in the state of the art, many new technologies have offered solutions to fulfill the needs of autistic individuals. Notably, several studies were focused on finding wearable assisting technologies. The main idea is to correct the stereotyped behavior of autistic individuals. Other studies have investigated urban planning solutions, consisting of adapting the public environment to ease autistic people's needs.

Lastly, there have been many attempts to implement guiding solutions. One such solution is implementing an application on the user’s phone that suggests a route according to specific criteria defined by the user. For instance, the user can select a route with fewer people or less noise. The idea is to help autistic individuals locate themselves in the area and guide them by giving them information on their phones. However, those solutions do not provide concrete applications, and they do not answer issues with accessibility within the application to make standards understandable.

Starting from here, the goal of this paper is to present the application we developed. Its purpose is to be used in real situations to guide autistic people, mainly by helping them make informed choices based on public transportation data that is relevant to them. We have mainly focused on the Toulouse area.

First, we will discuss the implementation of our algorithm to guide individuals in the city, using the OpenStreetMap database. Then we will present the inclusion of parameters that may affect autistic people and how we use them to influence the generated path.
