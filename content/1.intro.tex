\section{Introduction}


Nowadays, with the increasing growth in population, mobility has become a constant axis of development. However, mobility fails to address the issue of inclusivity. The infrastructures are not built to address the needs of people with disabilities or disorders. More precisely, individuals with Autism Spectrum Disorder (ASD), may suffer during transportation, due to visual or auditory overstimulation, or because of crowded areas.


Although this is a well-defined field of research, this paper should be seen as a contribution to the debate around understanding disabilities as a product of processes of human-environment interaction, an attempt to address mobility issues faced by people with disabilities by taking into account their challenges and capabilities.


As we have seen in the state of the art, many new technologies have offered possibilities to answer the needs of autistic individuals. Notably, several studies were focused on finding wearable assisting technologies. The main idea is to correct the stereotyped behavior of autistic individuals. Other studies have investigated urban planning solutions, consisting of adapting the public environment to ease autistic people's needs.


Lastly, there have been many attempts to implement guiding solutions. One solution is implemented by an application on the user’s phone that suggests a route according to specific criteria defined by the user. For instance, the user can select a route with fewer people or less noise. The idea is to help autistic individuals locate themselves in the area and guide them by giving them information on their phones. However, those solutions do not provide concrete applications, and they do not answer issues with accessibility within the application to make standards understandable. They are also limited to a defined area and cannot be used everywhere.


Starting from here, the goal of this paper is to present the application we developed to be used in real situations. The purpose is to guide autistic people, mainly by helping them make informed choices based on public transportation data that is relevant to them. We have mainly focused on the Toulouse area.


First, we will discuss the implementation of our algorithm to guide individuals in the city, using the OpenStreetMap database. Then we will present the inclusion of parameters that may affect autistic people and how we use them to influence the generated path.
