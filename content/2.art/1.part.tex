\subsection{Autism, a complex condition}

As defined by the DSM-5, autism spectrum disorder is a complex neurodevelopmental condition that manifests itself in a wide range of challenges related to social interaction, communication, repetitive behaviors, and often, restrictive interests.

The World Health Organization estimates that approximately 1 in 100 children worldwide has ASD. However, this figure may not capture the full extent of ASD prevalence due to underdiagnoses, which we will further explain in the report. It's important to note that an observed increase in ASD prevalence over time may be attributed, at least in part, to factors such as improved awareness, expanded diagnostic criteria, and increased access to services rather than a true rise in the prevalence of the condition.

Concerning the official diagnosis, there are several recurring themes. First and foremost, autistic individuals often struggle with social interactions. They may find it difficult to initiate or sustain conversations, understand non-verbal cues like facial expressions and body language, and may have trouble forming and maintaining relationships. This comes hand in hand with communication challenges, which can vary widely. Some individuals may have delayed speech development or may never develop spoken language. Others may have a rich vocabulary but struggle with pragmatic aspects of communication, such as understanding sarcasm or maintaining appropriate eye contact.

Routine is of great importance to a lot of autistic individuals. It helps maintain a sense of safety in a constantly unsafe world. Furthermore, many engage in what is known as stimming, or self-stimulatory behavior (like rocking, hand-flapping, and nail-biting), that helps them soothe emotional and/or sensory overloads. On that note, autistic individuals may be hypersensitive or hyposensitive to sensory stimuli. They may find certain sounds, textures, or lights overwhelming, while others may not register sensory input as much as expected.

Another behavior that characterizes ASD is special interests, referring to a particular topic, subject, or activity that captures someone's attention and enthusiasm to a significant degree. The number of such interests varies from person to person. Some may stay engrossed by the same topic for years, while others may jump from special interest to special interest every few months. These special interests often involve periods of hyper-focus, which result in a great expertise on said domain from the part of the autistic individual.

The autistic experience is not a monolith. It depends on many factors beyond the mere criteria of the diagnosis itself. To live as an autistic person is, before anything else, a personal experience that cannot be reduced to a series of symptoms.

\subsubsection{Co-occurring Conditions and Mental Health Challenges}

Over time, the diagnostic criteria for Autism Spectrum Disorder have evolved. The introduction of the Diagnostic and Statistical Manual of Mental Disorders, Fifth Edition (DSM-5) in 2013 marked a significant shift, consolidating previous separate diagnoses like autistic disorder and Asperger's syndrome into a unified diagnosis of ASD. This transition has had notable effects on prevalence rates and diagnostic approaches. Despite progress, research on ASD remains relatively recent, indicating that much remains to be discovered about the intricacies of this condition.

Autism is not classified as a mental health disorder, yet a significant proportion of individuals on the autism spectrum grapple with mental health challenges. Elevated rates of anxiety and depression among individuals with autism have been linked to various adverse outcomes, including diminished life satisfaction, heightened social challenges, feelings of loneliness, and difficulties with sleep, such as insomnia. These struggles can have profound and devastating consequences. Studies indicate that individuals on the autism spectrum are considerably more prone to contemplating, attempting, and tragically, dying by suicide compared to the general population.

ASD often co-occurs with other developmental and psychiatric conditions.
Specifically, 84.1\% of the ASD children met DSM-III-R criteria for at least one anxiety disorder, 25\% met criteria for Obsessive compulsive disorder\cite{2009AnxietyChildrenAdolescents}, 29.2\% suffer from social anxiety disorder, 28\% have ADHD diagnosis\cite{2008PsychiatricDisordersChildren} Furthermore, "some genetic disorders are more common in children with ASD such as Fragile X syndrome, Down syndrome, Duchenne muscular dystrophy, neurofibromatosis type I, and tuberous sclerosis complex. Children with autism are also more prone to a variety of neurological disorders, including epilepsy, macrocephaly, hydrocephalus, cerebral palsy, migraine/headaches, and congenital abnormalities of the nervous system."\cite{2021AutismMedicalComorbidities}

\subsubsection{Recognizing Biases and Addressing Disparities in Healthcare}

Autism is a complex spectrum disorder characterized by a diverse array of symptoms and severity levels among affected individuals. This inherent variability poses considerable challenges in establishing clear diagnostic criteria, often resulting in misdiagnosis or delayed diagnosis. While the precise etiology of autism remains elusive, researchers suggest a multifaceted interplay of genetic and environmental factors contributing to its development. Consequently, the presentation of autism is highly heterogeneous, further complicating diagnostic efforts.

One of the fundamental difficulties in diagnosing autism lies in the overlap of its symptoms with those of other developmental disorders. Distinguishing between these conditions can be arduous, leading to diagnostic confusion and the co-occurrence of multiple conditions in affected individuals. Additionally, traditional diagnostic criteria have primarily been derived from research focused on male populations, potentially overlooking or misidentifying autism in females and individuals from diverse cultural backgrounds. Recognizing this bias is crucial in ensuring equitable access to diagnosis and support services.

Gender disparities in autism diagnosis are quite striking. The female-male diagnostic discrepancy underscores the importance of adopting culturally sensitive and gender-inclusive diagnostic approaches to capture the full spectrum of autism presentations accurately. Studies have indicated that females with autism may exhibit different symptom profiles or employ coping mechanisms such as masking to camouflage their autistic traits, further complicating diagnosis and intervention efforts.

Furthermore, disparities in autism diagnosis extend beyond gender to encompass race, ethnicity, and socioeconomic status. Research has shown that children from minority and economically disadvantaged backgrounds may experience delayed diagnosis or reduced access to support services, perpetuating systemic inequalities in healthcare provision. Cultural stigmas surrounding neurodiversity and mental health also play a significant role, influencing individuals and families' willingness to seek diagnosis and support.

Despite the progress in the medical domain concerning autism, certain phenomena remain poorly researched. Autistic masking, for instance, is a very common phenomenon observed in individuals with autism wherein they suppress or conceal their autistic traits in social situations to blend in or appear more socially adept. While masking may facilitate social interactions and mitigate stigma, it can exact a toll on mental well-being, contributing to heightened stress, anxiety, and feelings of alienation. Research suggests that masking is more prevalent among females with autism and may contribute to underdiagnosis or misdiagnosis, highlighting the need for greater awareness and understanding of this phenomenon\cite{2017PuttingMyBest}.

Addressing the challenges associated with autism diagnosis and support requires a multifaceted approach that acknowledges the diverse needs and experiences of individuals across different demographic groups. Failure to recognize and address these disparities may perpetuate inequities in healthcare provision and exacerbate the impact of autism on affected individuals' lives.
